\documentclass[12pt, a4paper]{ctexart} % 使用ctexart文档类支持中文
\usepackage[top=2.5cm, bottom=2.5cm, left=3cm, right=2cm]{geometry} % 设置页边距
\usepackage{amsmath} % 数学公式支持
\usepackage{graphicx} % 插入图片
\usepackage{listings} % 代码块支持
\usepackage{xcolor} % 颜色支持
\usepackage{hyperref} % 超链接支持

% 设置代码块样式(可根据需要调整)
\lstset{
    basicstyle=\small\ttfamily,
    numbers=left,
    numberstyle=\tiny,
    frame=shadowbox,
    breaklines=true,
    backgroundcolor=\color[rgb]{0.95,0.95,0.97}
}

\title{人工智能引论} % 主标题
\author{} % 作者留空(学号姓名放在小标题)
\date{} % 日期留空

\begin{document}

\maketitle % 生成标题

\vspace{-6.3em}

\section*{Homework } % 作业次数(修改数字即可)
\section*{内容:}
\section*{学号: \quad 姓名:} % 学号姓名(修改此处信息)

\vspace{+3.5em}

\begin{enumerate} % 开始题目列表

    \item \textbf{题目:}
    
    \textbf{解答:} 
    
    \item \textbf{题目:}
    
    \textbf{解答:} 

    \item \textbf{题目:}

    \textbf{解答:}

    \item \textbf{题目:}

    \textbf{解答:}

    \item \textbf{题目:}

    \textbf{解答:}

    \item \textbf{题目:}

    \textbf{解答:}

    \item \textbf{题目:}

    \textbf{解答:}
    
\end{enumerate}

% 插入图片示例(保持注释,需要时取消注释使用)
%\begin{figure}[h]
%    \centering
%    \includegraphics[width=0.8\textwidth]{ai.png}
%    \caption{人工智能发展示意图}
%\end{figure}

\end{document}